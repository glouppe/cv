%%%%%%%%%%%%%%%%%%%%%%%%%%%%%%%%%%%%%%%%%
% Classicthesis-Styled CV
% LaTeX Template
% Version 1.0 (22/2/13)
%
% This template has been downloaded from:
% http://www.LaTeXTemplates.com
%
% Original author:
% Alessandro Plasmati
%
% License:
% CC BY-NC-SA 3.0 (http://creativecommons.org/licenses/by-nc-sa/3.0/)
%
%%%%%%%%%%%%%%%%%%%%%%%%%%%%%%%%%%%%%%%%%

%----------------------------------------------------------------------------------------
%	PACKAGES AND OTHER DOCUMENT CONFIGURATIONS
%----------------------------------------------------------------------------------------

\documentclass{scrartcl}

\reversemarginpar
\newcommand{\MarginText}[1]{\marginpar{\raggedleft\itshape\small#1}}

\usepackage[nochapters]{classicthesis}
\usepackage[LabelsAligned]{currvita}

\usepackage[a4paper, lmargin=2in, rmargin=1.5in, tmargin=1.5in, bmargin=1.5in, footskip=0.5in]{geometry}

\renewcommand{\cvheadingfont}{\LARGE\color{Maroon}}

\usepackage{hyperref}
\hypersetup{colorlinks, breaklinks, urlcolor=Maroon, linkcolor=Maroon}

\newcommand{\NewEntry}[2]{\noindent\hangindent=0em\hangafter=0 \MarginText{#1}#2\vspace{0.5em}}
\newcommand{\NewEntryFull}[3]{\noindent\hangindent=0em\hangafter=0 #2\MarginText{#1}{\ \ $\cdotp$\ \ \it #3}\vspace{0.5em}}
\newcommand{\NewPublication}[4]{\noindent\hangindent=0em\hangafter=0 \MarginText{\color{black} #1}{\footnotesize [{\color{Maroon}#2}]} #3 {\footnotesize\color{gray}#4}\vspace{0.5em}}
\newcommand{\Description}[1]{\hangindent=2em\hangafter=0\noindent\raggedright\footnotesize{#1}\par\normalsize\vspace{1em}}
\newcommand{\DescriptionAligned}[1]{\hangindent=0em\hangafter=0\noindent\raggedright{#1}\par\normalsize\vspace{1em}}


%----------------------------------------------------------------------------------------

\usepackage{scrpage2}

\deftripstyle{cv-headings}%
{}{}{Gilles \textsc{Louppe} -- Academic resume}%
{}{}{}


\date{}

\begin{document}

\pagestyle{cv-headings}

% \pagestyle{empty}
%\pagestyle{scrheadings}

%\thispagestyle{empty}

%----------------------------------------------------------------------------------------
%	NAME AND CONTACT INFORMATION SECTION
%----------------------------------------------------------------------------------------

\begin{cv}{\spacedallcaps{Academic curriculum}}\vspace{3em}

\noindent\spacedlowsmallcaps{Personal Information}\vspace{1em}

\NewEntry{name}{Gilles Louppe}{}

\NewEntry{birth date}{Born in Belgium, on April 26, 1987.}{}

\NewEntry{office}{CERN\\
Office 32-S-A19\\
CH-1211 Geneva\\
Switzerland}

\NewEntry{home}{Route de Marnex, 22a B2\\
CH-1291 Commugny\\
Switzerland}

\NewEntry{phone}{+41 (0) 77 99 53 433}

\NewEntry{email}{\href{mailto:g.louppe@nyu.edu}{g.louppe@nyu.edu}}

\NewEntry{web}{\url{https://glouppe.github.io}}

\NewEntry{scholar}{\href{https://scholar.google.be/citations?user=F_77d4QAAAAJ}{F\_77d4QAAAAJ}}

\NewEntry{orcid}{\href{http://orcid.org/0000-0002-2082-3106}{0000-0002-2082-3106}}

\NewEntry{github}{\href{https://github.com/glouppe}{@glouppe}}

\vspace{2em}

\noindent\spacedlowsmallcaps{Research interests and objectives}\vspace{1em}

\NewEntry{}{%
As a researcher, my far ambition is to unlock discoveries currently beyond reach by
making statistical inference, machine learning and artificial intelligence
a cornerstone of the modern scientific method. Using particle physics as a test bed, my present research
interests circle around how to use or design new machine learning algorithms to
approach data-driven scientific problems in new and transformative ways. With
this goal in mind, my topics of research include methods for simulator-based
likelihood-free inference, algorithms to handle systematic uncertainties in
inference models, and developments towards the automation of science.
}\vspace{2em}


%----------------------------------------------------------------------------------------
%	WORK EXPERIENCE
%----------------------------------------------------------------------------------------

\noindent\spacedlowsmallcaps{Work Experience}\vspace{1em}

\NewEntryFull{2017--Present}{Assistant Professor}{University of Li{\`e}ge (Belgium)}

\Description{Research in machine learning and artificial intelligence.}
\Description{Chair of Big Data, with financial support from NRB.}

\NewEntryFull{2015--2017}{Postdoctoral Associate}{New York University (USA), CERN (Switzerland)}

\Description{Machine learning and artificial intelligence for Science.}
\Description{Data Science Fellow for the Moore-Sloan Data Science Environment.\\
             Research Scientist at CERN.}

\NewEntryFull{2014--2015}{Marie-Curie COFUND Research Fellow}{CERN (Switzerland)}

\Description{Machine learning and data analysis on text and bibliographic data.}

\NewEntryFull{2010--2014}{F.R.S.-FNRS Research Fellow}{University of Li{\`e}ge (Belgium)}

\Description{Fundamental and applied research in machine learning and data analysis. Expertise in tree-based methods.}
\Description{Teaching assistant (Introduction to Algorithms, Data Structures and Algorithms, Machine Learning).}

\NewEntryFull{2007--2010}{Student instructor}{University of Li{\`e}ge (Belgium)}

\Description{Mentoring of undergraduate students (Introduction to Algorithms).}

\vspace{2em}


%----------------------------------------------------------------------------------------
%	EDUCATION
%----------------------------------------------------------------------------------------

\spacedlowsmallcaps{Education}\vspace{1em}

\NewEntryFull{2010--2014}{PhD in Computer Sciences}{University of Li{\`e}ge (Belgium)}

\Description{Thesis: Understanding Random Forests -- From Theory to Practice.}

\NewEntryFull{2008--2010}{Master in Computer Sciences}{University of Li{\`e}ge (Belgium)}

\Description{Thesis: Collaborative Filtering -- Scalable approaches using Restricted Boltzmann Machines.\newline
Erasmus student at the Royal Institute of Technology (KTH), Sweden.\newline
\textit{Summa cum laude.}}

\NewEntryFull{2005--2008}{Bachelor in Computer Sciences}{University of Li{\`e}ge (Belgium)}

\Description{\textit{Summa cum laude.}}

\vspace{2em}


%----------------------------------------------------------------------------------------
%	PUBLICATIONS
%----------------------------------------------------------------------------------------

\spacedlowsmallcaps{Publications}\vspace{1em}

\NewPublication{pre-prints}{4}{Adversarial Variational Optimization of Non-Differentiable Simulators.}{%
Gilles Louppe, Kyle Cranmer.
{\it To be submitted to a journal. Pre-print available.}
[\href{https://arxiv.org/abs/1707.07113}{PDF}]
{\color{black}\tt (a1)}}

\NewPublication{}{3}{QCD-aware Recursive Neural Networks for Jet Physics.}{%
Gilles Louppe, Kyunghyun Cho, Kyle Cranmer.
{\it To be submitted to a journal. Pre-print available.}
[\href{https://arxiv.org/abs/1702.00748}{PDF}]
{\color{black}\tt (a1)}}

\NewPublication{}{2}{Learning to Pivot with Adversarial Networks.}{%
Gilles Louppe, Michael Kagan, Kyle Cranmer.
{\it  Submitted to NIPS 2017. Pre-print available.}
[\href{https://arxiv.org/abs/1611.01046}{PDF}, \href{https://github.com/glouppe/paper-learning-to-pivot}{Code}]
{\color{black}\tt (p1)}}

\NewPublication{}{1}{Approximating Likelihood Ratios with Calibrated Discriminative Classifiers.}{%
Kyle Cranmer, Juan Pavez, Gilles Louppe.
{\it Submitted to Journal of the American Statistical Association. Pre-print available. }
[\href{http://arxiv.org/abs/1506.02169}{PDF}, \href{https://github.com/diana-hep/carl}{Code}]
{\color{black}\tt (a1)}}



\NewPublication{in journals}{5}{Collaborative analysis of multi-gigapixel imaging data using Cytomine.}{%
Rapha\"el Mar\'ee, Lo\"ic Rollus, Benjamin St\'evens, Renaud Hoyoux, Gilles Louppe, Remy Vandaele, Jean-Michel Begon, Philipp Kainz, Pierre Geurts, Louis Wehenkel.
Bioinformatics, 32 (9), 1395--1401, 2016.
[\href{http://bioinformatics.oxfordjournals.org/content/early/2016/01/09/bioinformatics.btw013.full.pdf+html}{PDF}, \href{http://www.cytomine.be/}{Code}]
{\color{black}\tt (a1)}}

\NewPublication{}{4}{Scikit-learn: Machine Learning Without Learning the Machinery.}{%
Gael Varoquaux, Lars Buitinck, Gilles Louppe, et al.
GetMobile: Mobile Computing and Communications 19 (1), 29-33, 2015.
[\href{https://dl.acm.org/citation.cfm?id=2786995}{PDF}]
{\color{black}\tt (a2)}}

\NewPublication{}{3}{Solar Energy Prediction: An International Contest to Initiate Interdisciplinary Research on Compelling Meteorological Problems.}{%
Amy McGovern, David John Gagne II, Lucas Eustaquio, Gilberto Titericz Junior, Benjamin Lazorthes, Owen Zhang, Gilles Louppe, Peter Prettenhofer, Jeffrey Basara, Thomas Hamill, David Margolin.
Bulletin of the American Meteorological Society, 2015.
[\href{http://hdl.handle.net/2268/177115}{PDF}]
{\color{black}\tt (a2)}}

\NewPublication{}{2}{Exploiting SNP Correlations within Random Forest for Genome-Wide Association Studies.}{%
Vincent Botta, Gilles Louppe, Pierre Geurts, Louis Wehenkel.
PLoS ONE 9(4), 2014.
[\href{http://dx.plos.org/10.1371/journal.pone.0093379}{PDF}, \href{https://github.com/0asa/TTree-source}{Code}]
{\color{black}\tt (a1)}}

\NewPublication{}{1}{Scikit-Learn: Machine Learning in Python.}{%
Fabian Pedregosa, Ga\"el Varoquaux, Alexandre Gramfort, Vincent Michel, Bertrand Thirion, Olivier Grisel, Mathieu Blondel, Gilles Louppe, Peter Prettenhofer, Ron Weiss, Vincent Dubourg, Jake Vanderplas, Alexandre Passos, David Cournapeau, Matthieu Brucher, Matthieu Perrot, \'Edouard Duchesnay.
Journal of Machine Learning Research, 12, 2825--2830, 2012.
Last updated on January 10, 2017.
[\href{https://arxiv.org/abs/1201.0490}{PDF}, \href{https://http://scikit-learn.org}{Code}]
{\color{black}\tt (a1)}}



\NewPublication{in conference proceedings}{13}{Experiments using machine learning to approximate likelihood ratios for mixture models.}{%
Kyle Cranmer, Juan Pavez, Gilles Louppe, W. K. Brooks.
Journal of Physics: Conference Series, 2016.
[\href{http://iopscience.iop.org/article/10.1088/1742-6596/762/1/012034/pdf}{PDF}]
{\color{black}\tt (p1)}}

\NewPublication{}{12}{Random subspace with trees for feature selection under memory constraints.}{%
Antonio Sutera, C\'elia Chatel, Gilles Louppe, Louis Wehenkel, Pierre Geurts.
The 25th Belgian-Dutch Conference on Machine Learning, Leuven, Belgium, 2016.
[\href{http://hdl.handle.net/2268/202206}{PDF}]
{\color{black}\tt (c1)}}

\NewPublication{}{11}{Context-dependent feature analysis with random forests.}{%
Antonio Sutera, Gilles Louppe, V\^an Anh Huynh-Thu, Louis Wehenkel, Pierre Geurts.
UAI, New York, USA, 2016.
[\href{https://arxiv.org/abs/1605.03848}{PDF}]
{\color{black}\tt (p1)}}

\NewPublication{}{10}{Visualization of Publication Impact.}{%
Eamonn Maguire, Javier Martin Montull, Gilles Louppe.
EuroVis, Groningen, the Netherlands, 2016.
[\href{https://arxiv.org/abs/1605.06242}{PDF}, \href{https://github.com/inspirehep/impact-graphs}{Code}]
{\color{black}\tt (p1)}}

\NewPublication{}{9}{Ethnicity sensitive author disambiguation using semi-supervised learning.}{%
Gilles Louppe, Hussein Al-Natsheh, Mateusz Susik, Eamonn Maguire.
7th International Conference, KESW, Prague, Czech Republic, 2016. {\it Best paper award}.
[\href{http://arxiv.org/abs/1508.07744}{PDF}, \href{https://github.com/glouppe/paper-author-disambiguation/}{Code}]
{\color{black}\tt (p1)}}

\NewPublication{}{8}{Collaborative analysis of gigapixel images using Cytomine.}{%
R\'emy Vandaele, Rapha\"el Mar\'ee, Pierre Geurts, Lo\"ic Rollus, Benjamin St\'evens, Renaud Hoyoux, Jean-Michel Begon, Gilles Louppe, Louis Wehenkel.
Acta Stereologica, July, 2015.
[\href{http://popups.ulg.ac.be/0351-580X/index.php?id=3692&file=1&pid=3681}{PDF}]
{\color{black}\tt (c1)}}

\NewPublication{}{7}{Simple connectome inference from partial correlation statistics in calcium imaging.}{%
Antonio Sutera, Arnaud Joly, Vincent Francois-Lavet, Zixiao Aaron Qiu, Gilles Louppe, Damien Ernst, Pierre Geurts.
JMLR: Workshop and Conference Proceedings, 2014.
[\href{http://hdl.handle.net/2268/169767}{PDF}, \href{https://github.com/glouppe/kaggle-connectomics}{Code}]
{\color{black}\tt (p1)}}

\NewPublication{}{6}{A hybrid human-computer approach for large-scale image-based measurements using web services and machine learning.}{%
Rapha\"el Mar{\'e}e, Lo\"ic Rollus, Benjamin St\'evens, Gilles Louppe, et al.
11th IEEE International Symposium on Biomedical Imaging, Beijing, China, 2014.
[\href{http://hdl.handle.net/2268/162084}{PDF}]
{\color{black}\tt (p1)}}

\NewPublication{}{5}{Understanding variable importances in forests of randomized trees.}{%
Gilles Louppe, Louis Wehenkel, Antonio Sutera, Pierre Geurts.
NIPS, Lake Tahoe, United States, 2013.
[\href{http://hdl.handle.net/2268/155642}{PDF}, \href{http://github.com/glouppe/paper-variable-importances}{Code}]
{\color{black}\tt (p1)}}

\NewPublication{}{4}{API design for machine learning software: experiences from the scikit-learn project.}{%
Lars Buitinck, Gilles Louppe, Mathieu Blondel, et al.
ECML-PKDD 2013 Workshop: Languages for Data Mining and Machine Learning, Pragues, Czech Republic, 2013.
[\href{http://hdl.handle.net/2268/154357}{PDF}, \href{http://github.com/scikit-learn/scikit-learn}{Code}]
{\color{black}\tt (p1)}}

\NewPublication{}{3}{Ensembles on Random Patches.}{%
Gilles Louppe, Pierre Geurts.
ECML-PKDD 2012, Bristol, UK, 2012.
[\href{http://hdl.handle.net/2268/130099}{PDF}, \href{http://github.com/scikit-learn/scikit-learn/blob/master/sklearn/ensemble/bagging.py}{Code}]
{\color{black}\tt (p1)}}

\NewPublication{}{2}{Learning to rank with extremely randomized trees.}{%
Pierre Geurts, Gilles Louppe.
JMLR: Workshop and Conference Proceedings, 14, 49-61, 2011.
[\href{http://hdl.handle.net/2268/84538}{PDF}]
{\color{black}\tt (p1)}}

\NewPublication{}{1}{A zealous parallel gradient descent algorithm.}{%
Gilles Louppe, Pierre Geurts.
Learning on Cores, Clusters and Clouds workshop, NIPS, Vancouver, Canada, 2010.
[\href{http://hdl.handle.net/2268/80780}{PDF}, \href{http://hdl.handle.net/2268/80780}{Code}]
{\color{black}\tt (c1)}}



\NewPublication{thesis}{2}{Understanding Random Forests: From Theory to Practice.}{%
Gilles Louppe.
PhD thesis, University of Li{\`e}ge, 2014.
[\href{http://hdl.handle.net/2268/170309}{PDF}, \href{https://github.com/glouppe/phd-thesis}{Code}]}

\NewPublication{}{1}{Collaborative filtering: Scalable approaches using restricted Boltzmann machines.}{%
Gilles Louppe.
Master's thesis, University of Li{\`e}ge, 2010.
[\href{http://hdl.handle.net/2268/74400}{PDF}, \href{http://hdl.handle.net/2268/74400}{Code}]}



\NewPublication{others}{1}{Pitfalls of evaluating a classifier’s performance in high energy physics applications.}{%
Gilles Louppe, Tim Head.
[\href{http://dx.doi.org/10.5281/zenodo.34934}{Notebook}]}




\vspace{2em}


%----------------------------------------------------------------------------------------
%   TALKS
%----------------------------------------------------------------------------------------

\spacedlowsmallcaps{Scholar activities}\vspace{1em}

\NewPublication{talks}{31}{Adversarial Variational Optimization of Non-Differentiable Simulators.}{%
Hammer and Nails - Machine Learning and HEP.
July 19-28, 2017.
[\href{https://www.weizmann.ac.il/conferences/SRitp/Summer2017/program}{Materials}]}

\NewPublication{}{30}{Teaching machines to discover particles.}{%
Hammer and Nails - Machine Learning and HEP.
July 19-28, 2017.
[\href{https://www.weizmann.ac.il/conferences/SRitp/Summer2017/program}{Materials}]}

\NewPublication{}{29}{Learning to Pivot with Adversarial Networks.}{%
Data Science @ HEP 2017, Fermilab, USA.
May 8-12 2017.
[\href{https://github.com/glouppe/talk-learning-to-pivot}{Materials}]}

\NewPublication{}{28}{Learning to Pivot with Adversarial Networks.}{%
IIHE, Brussels, Belgium.
April 7, 2017.
[\href{https://github.com/glouppe/talk-learning-to-pivot}{Materials}]}

\NewPublication{}{27}{Learning to Pivot with Adversarial Networks.}{%
IML Forum, CERN, Switzerland.
December 15, 2016.
[\href{https://github.com/glouppe/talk-learning-to-pivot}{Materials}]}

\NewPublication{}{26}{Learning to Pivot with Adversarial Networks.}{%
ATLAS ML Forum, CERN, Switzerland.
November 17, 2016.
[\href{https://github.com/glouppe/talk-learning-to-pivot}{Materials}]}

\NewPublication{}{25}{Learning to generate with adversarial networks.}{%
US ATLAS Physics Support, Software and Computing meeting, Chicago, USA.
August 3, 2016.
[\href{https://indico.cern.ch/event/526308/}{Materials}]}

\NewPublication{}{24}{Learning to generate with adversarial networks.}{%
ATLAS ML Forum, CERN, Switzerland.
July 21, 2016.
[\href{https://indico.cern.ch/event/556591/}{Materials}]}

\NewPublication{}{23}{Learning to generate with adversarial networks.}{%
DS @ HEP at the Simons Foundation, New York, USA.
July 5-7, 2016.
[\href{https://indico.hep.caltech.edu/indico/conferenceTimeTable.py?confId=102}{Materials}]}

\NewPublication{}{22}{Learning to generate with adversarial networks.}{%
Software Tech Forum, CERN, Switzerland.
June 27, 2016.
[\href{https://indico.cern.ch/event/544644/contributions/2210328/attachments/1299201/1938586/slides.pdf}{Materials}]}

\NewPublication{}{21}{Robust and Calibrated Classifiers with Scikit-Learn.}{%
Zurich ML meetup, Switzerland.
April 13, 2016.
[\href{https://github.com/glouppe/tutorials-scikit-learn}{Materials}]}

\NewPublication{}{20}{Approximating likelihood ratios with Calibrated Classifiers.}{%
ETH, Zurich, Switzerland.
April 13, 2016.
[\href{https://github.com/glouppe/talk-approximating-likelihood-ratios-with-classifiers}{Materials}]}

\NewPublication{}{19}{Approximating likelihood ratios with Calibrated Classifiers.}{%
ATLAS ML workshop, CERN, Switzerland.
March 29-31, 2016.
[\href{https://github.com/glouppe/talk-approximating-likelihood-ratios-with-classifiers}{Materials}]}

\NewPublication{}{18}{An introduction to Bayesian Optimization.}{%
ATLAS ML workshop, CERN, Switzerland.
March 29-31, 2016.
[\href{https://github.com/glouppe/talk-bayesian-optimisation}{Materials}]}

\NewPublication{}{17}{Pitfalls of evaluating a classifier’s performance in high energy physics applications.}{%
ALEPH workshop, NIPS, Montr\'eal, Canada.
December 11, 2015.
[\href{https://github.com/glouppe/talk-aleph-workshop2015}{Materials}]}

\NewPublication{}{16}{Classification with a control channel: Don't cheat yourself!}{%
CERN, Switzerland.
October 5, 2015.
[\href{https://github.com/glouppe/talk-classification-control-channel}{Materials}]}

\NewPublication{}{15}{Understanding Random Forests.}{%
CERN, Switzerland.
September 21, 2015.
[\href{https://github.com/glouppe/talk-pydata2015}{Materials}]}

\NewPublication{}{14}{Tree models with Scikit-Learn: Great learners with little assumptions.}{%
PyData, Paris, France.
April 5, 2015.
[\href{https://github.com/glouppe/talk-pydata2015}{Materials}]}

\NewPublication{}{13}{Machine Learning for Author Disambiguation.}{%
CERN, Switzerland.
March 3, 2015.
[\href{https://github.com/glouppe/talk-disambiguation-inspire}{Materials}]}

\NewPublication{}{12}{Bias-variance decomposition in Random Forests.}{%
Paris Machine Learning Meetup 4 (saison 2), Paris, France.
December 9, 2014.
[\href{http://hdl.handle.net/2268/174897}{Materials}]}

\NewPublication{}{11}{Scikit-Learn in Particle Physics.}{%
Data Science Academic software: From scikit-learn and scikit-image to domain science, Paris, France.
November 18, 2014.
[\href{https://github.com/glouppe/talk-cds2014}{Materials}]}

\NewPublication{}{10}{Understanding Random Forests: From Theory to Practice.}{%
Li{\`e}ge, Belgium.
October 9, 2014.
[\href{https://github.com/glouppe/phd-thesis}{Materials}]}

\NewPublication{}{9}{Accelerating Random Forests in Scikit-Learn.}{%
EuroScipy, Cambridge, UK.
August 29, 2014.
[\href{https://github.com/glouppe/talk-euroscipy2014}{Materials}]}

\NewPublication{}{8}{Gradient Boosted Regression Trees in Scikit-Learn.}{%
PyData, London, UK.
February 23, 2014.
[\href{https://github.com/glouppe/tutorial-pydata2014}{Materials}]}

\NewPublication{}{7}{Forecasting Daily Solar Energy Production Using Robust Regression Techniques.}{%
Atlanta, USA.
February 5, 2014.
[\href{http://hdl.handle.net/2268/162797}{Materials}]}

\NewPublication{}{6}{Scikit-Learn: Machine Learning in the Python ecosystem.}{%
NIPS Workshop on Machine Learning Open Source Software,
Lake Tahoe, USA.
December 10, 2013.
[\href{http://hdl.handle.net/2268/157487}{Materials}]}

\NewPublication{}{5}{Understanding variable importances in forests of randomized trees.}{%
NIPS,
Lake Tahoe, USA.
December 8, 2013.
[\href{http://hdl.handle.net/2268/155642}{Materials}]}

\NewPublication{}{4}{Scikit-Learn, or why I joined an open source software project.}{%
University of Li{\`e}ge, Belgium.
October 30, 2013.
[\href{http://www.slideshare.net/glouppe/scikitlearn-or-why-i-joined-an-open-source-software-project}{Materials}]}

\NewPublication{}{3}{Ensembles on Random Patches.}{%
ECML,
Bristol, UK.
September 25, 2012.
[\href{http://hdl.handle.net/2268/130099}{Materials}]}

\NewPublication{}{2}{Large-scale machine learning for collaborative filtering.}{%
Groupe de contact FNRS Calcul Intensif,
University of Li{\`e}ge, Belgium.
April 28, 2011.}

\NewPublication{}{1}{A zealous parallel gradient descent algorithm.}{%
NIPS Workshop on Learning on Cores, Clusters and Clouds,
Whistler, Canada.
December 11, 2010.
[\href{http://hdl.handle.net/2268/80780}{Materials}]}



\NewPublication{invited lectures}{7}{An introduction to machine learning with Scikit-Learn.}{%
IML Workshop, CERN, Switzerland
March 21, 2017.
[\href{https://github.com/glouppe/tutorials-iml2017}{Materials}]}

\NewPublication{}{6}{Series of Lectures on Machine Learning.}{%
Machine Learning and Data Science in Physics, Barcelona
October 17-21, 2016.
[\href{https://github.com/iccub-ml/lectures-glouppe}{Materials}]}

\NewPublication{}{5}{Approximating likelihood ratios with Calibrated Classifiers.}{%
2nd MLHEP summer school, Lund, Sweden
June 22, 2016.
[\href{https://github.com/glouppe/talk-approximating-likelihood-ratios-with-classifiers}{Materials}]}

\NewPublication{}{4}{An introduction to machine learning with Scikit-Learn.}{%
Heavy Flavour Data Mining workshop, Zurich, Switzerland
February 18, 2016.
[\href{https://github.com/glouppe/tutorial-scikit-learn}{Materials}]}

\NewPublication{}{3}{An introduction to machine learning with Scikit-Learn.}{%
Data Science at LHC, Switzerland.
November 12, 2015.
[\href{https://github.com/glouppe/tutorial-sklearn-dslhc2015}{Materials}]}

\NewPublication{}{2}{An introduction to machine learning with Scikit-Learn.}{%
AstroHack Week, New York, USA.
September 30, 2015.
[\href{https://github.com/AstroHackWeek/AstroHackWeek2015}{Materials}]}

\NewPublication{}{1}{An introduction to Machine Learning with Scikit-Learn.}{%
CERN, Switzerland.
April 23, 2015.
[\href{https://github.com/glouppe/tutorial-sklearn-lhcb}{Materials}]}




\vspace{2em}



%----------------------------------------------------------------------------------------
%   Supervision
%----------------------------------------------------------------------------------------

\noindent\spacedlowsmallcaps{Academic Supervision}\vspace{1em}

\NewEntryFull{2017}{Paul Klein (MSc student)}{Supelec (France)}

\Description{CERN intern. Generative models for particle detectors.}

\NewEntryFull{2016}{Manoj Kumar (MSc student)}{New York University (USA)}

\Description{Junior data scientist at the Center for Data Science (New York).
Sequential model-based optimization.}

\NewEntryFull{2014--2015}{Mateusz Susik (MSc student)}{University of Warsaw (Poland)}

\Description{CERN intern.
Author disambiguation with supervised learning.}

\NewEntryFull{2014--2015}{Hussein Al-Natsheh (MSc student)}{University of Lyon 2 (France)}

\Description{CERN intern.
Author disambiguation with supervised learning.}

\NewEntryFull{2014--2015}{Joseph Boyd (MSc student)}{EPFL (Switzerland)}

\Description{CERN intern.
Text mining with Machine Learning.}

\vspace{2em}



\noindent\spacedlowsmallcaps{Academic Service}\vspace{1em}

\NewEntry{as a peer reviewer}{Journals: Bioinformatics,
                                        Journal of the American Statistical Association,
                                        Journal of Machine Learning Research,
                                        Machine Learning,
                                        Neurocomputing.}

\NewEntry{}{Conferences: ICML 2016, ICML 2017,
                         NIPS 2014, NIPS 2015, NIPS 2016, NIPS 2017.}

\NewEntry{as a PC member}{Data Science @ HEP at the Simons Foundation, 2016.\\
                                         Data Science @ LHC Workshop, 2015.}

\vspace{2em}


%----------------------------------------------------------------------------------------
%   Software
%----------------------------------------------------------------------------------------

\noindent\spacedlowsmallcaps{Open Source Software}\vspace{1em}


\NewEntryFull{2016--Present}{Scikit-Optimize}{\url{https://scikit-optimize.github.io}}

\Description{Founder of Scikit-Optimize, a library for sequential model-based optimization.}

% \NewEntryFull{2016}{Carl}{\url{http://diana-hep.org/carl/}}
%
% \Description{Founder of Carl a toolbox for likelihood-free inference in Python.}

\NewEntryFull{2011--Present}{Scikit-Learn}{\url{https://scikit-learn.org}}

\Description{Core developer of Scikit-Learn, a widely adopted machine learning library.}


\vspace{2em}


% %----------------------------------------------------------------------------------------
% %	COMPUTER SKILLS
% %----------------------------------------------------------------------------------------
%
% \spacedlowsmallcaps{Skills}\vspace{1em}
%
% \DescriptionAligned{\MarginText{Specialties}Machine Learning, Statistics, Scientific Computing}
%
% \DescriptionAligned{\MarginText{Programming}\textsc{Python}, \textsc{C}}
%
% \vspace{1em}


%----------------------------------------------------------------------------------------
%	OTHER INFORMATION
%----------------------------------------------------------------------------------------

\spacedlowsmallcaps{Awards}\vspace{1em}

\NewEntry{2016}{Best paper award for {\it Ethnicity sensitive author disambiguation using
semi-supervised learning} at KESW'2016}

\NewEntry{2015}{AIM Prize for best PhD's thesis}

\NewEntry{2010--2014}{F.R.S.-FNRS research fellow scholarship}

\NewEntry{2010}{Melchior Salier Award for best Master's Thesis}

\NewEntry{2010}{Baudouin Elleboudt Award for best Master's Thesis}

\vspace{2em}

%------------------------------------------------


\spacedlowsmallcaps{Languages}\vspace{1em}

\newlength{\langbox}
\settowidth{\langbox}{English}

\DescriptionAligned{\parbox{\langbox}{\textsc{French}}\ \ $\cdotp$\ \ \ Mothertongue}

\vspace{-0.5em}

\DescriptionAligned{\parbox{\langbox}{\textsc{English}}\ \ $\cdotp$\ \ \ Fluent}

\vspace{-0.5em}

\DescriptionAligned{\parbox{\langbox}{\textsc{Dutch}}\ \ $\cdotp$\ \ \ Basic}

\vspace{1em}

% %------------------------------------------------
%
% \DescriptionAligned{\MarginText{Interests}Rowing\ \ $\cdotp$\ \ Cinema\ \ $\cdotp$\ \ Travel\ \ $\cdotp$\ \ Coffee\ \ $\cdotp$\ \ Vinyls}
%

\end{cv}
\end{document}
